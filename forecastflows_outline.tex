\documentclass[11pt]{article}
\usepackage[top=1.00in, bottom=1.0in, left=1in, right=1in]{geometry}
\renewcommand{\baselinestretch}{1.1}
\usepackage{graphicx}
\usepackage{natbib}
\usepackage{amsmath}
\usepackage{hyperref}

\begin{document}
\renewcommand{\refname}{\CHead{}}

\title{Closing the gap between statistical and scientific workflows for improved forecasts in ecology } 
\date{\today}
\author{Victor Van der Meersch, J. Regetz, T. J. Davies$^*$ \& EM Wolkovich}
\maketitle
$^*$ Says he is happy to help and give friendly review, but not sure he will reach level of co-author. 

\begin{enumerate}
\item Intro
\begin{itemize}
\item Ecology super challenged to predict stuff for decision making (kind of a whole new world of relevance?)\\
$\rightarrow$ Example of stuff to predict: populations, policy-relevant questions...
\item General way to do this so far... (bifurcated?)
\begin{itemize}
\item long term data (GBIF, Biotime...) for estimation of trends
\item PBM, SDMs for forecast
\end{itemize}
\item Gap, problem: None of this is going well
\begin{itemize}
\item debates over trends, not only on the significance but also the direction!
\item predictive modeling trapped in overly complex models, hard to get scientific insights? (same as GCMs?)
\end{itemize}
\item Here we introduce a universal workflow to adress this (say more)
\end{itemize}
\item Overview
\begin{itemize}
\item General scientific method we all learn stresses: RQ $\rightarrow$ study design $\rightarrow$ collect data $\rightarrow$ build model $\rightarrow$ answer
\item Divergences from this are common: 
\begin{itemize}
\item bad science (data lead to question)
\item important question for which we cannot get data quickly, we have to use existing
\item complexity makes this simple workflow hard/impossible
\end{itemize}
\item this workflow works across these realities by
\begin{itemize}
\item stressing the need to think about model before study design
\item advancing data simulation
% \item (outcome: spending more time in current rare quadrats will improve forecasts!)
\end{itemize}
\item More details on ideal workflow:  walk through the different steps
\begin{itemize}
\item spend more time in critical quadrat, post-model pre-data
\item feedbacks
\item uncertainty
\end{itemize}
\end{itemize}
\item How to address current issues (two case studies)
\begin{enumerate}
\item Trends!
\begin{itemize}
\item Outline current problem: different answers from different analysis (and slightly different datasets?)\\
$\rightarrow$ It is a problem because we can't make decisions on +/- debate... (and it degrades trust in science?)
\item two big missing parts are data simulation steps:
\begin{itemize}
\item retrodictive check: would highlight missing pieces (speed up current process, because without this step each slightly different model is a paper... whereas each big iteration of the workflow---including multiple feedbacks---should be a paper)
\item simulated data: you would know that you have low data sooner => the debate is actually not jus models but really limited data
\end{itemize}
\end{itemize}
\item Forecasts!
\begin{itemize}
\item Outline current problem:
\begin{itemize}
\item black box: intrication between model build and data fitting (calibration), everything is mixed 
\item complexity trap [mention uncertainty]
\item developping a model has become the goal, whereas it should be a way to answer a research question!
\end{itemize}
\item We need the workflow to open the black box! Simulating data would allow to add a necessary step between model building and data fitting, which would highlight strong degeneracies
\item Workflow would also force you to clearly express a research question, define a limited context in which the model should apply
\end{itemize}
\item Step back
\begin{itemize}
\item we need more data, and better question (relate this to both previous study cases)
\item where can we best reduce uncertainties through new scientific insights?
\item machine learning! If we do nothing, what's the point of not doing ML? ML $>$ process-based without question, and ML $>$ trends without mechanisms
\end{itemize}
\end{enumerate}
\item Wrap up: how to make it happen?
\begin{itemize}
\item usual issues: publication pressure, low standards (especially for models)\\
"\emph{we allow far more hand-waving in the presentation of modeling results than we do for experimental data}" (\href{https://harvardforest1.fas.harvard.edu/publications/pdfs/Aber_BulletinEcoSocAmerica_1997.pdf}{J. Aber, 1997})
\item growing concerns, leading to increase in reproducibility and data sharing practices [mention uncertainty]
\item need a little more here...
\item better training! on BOTH estimation AND prediction
\begin{itemize}
\item estimation: being aware of what is a parameter, and mention uncertainty propagation
\item prediction: should be a natural outcome, not a finality
\end{itemize}
\item ML, benchmarking models are probably useful but should not be the core of our scientific practice, not the spirit!
\end{itemize}
\end{enumerate}

\end{document}

