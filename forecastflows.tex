\documentclass[11pt]{article}
\usepackage[top=1.00in, bottom=1.0in, left=1in, right=1in]{geometry}
\renewcommand{\baselinestretch}{1.1}
\usepackage{graphicx}
\usepackage{natbib}
\usepackage{amsmath}
\usepackage{parskip}
\usepackage{todonotes}

\def\labelitemi{--}
\parindent=0pt

\begin{document}
\renewcommand{\refname}{\CHead{}}

\title{Closing the gap between statistical and scientific workflows for improved forecasts in ecology } 
\date{\today}
\author{Victor Van der Meersch, J. Regetz, T. J. Davies \& EM Wolkovich}\\
 \maketitle

{\bf Deadline:} 1 May 2025 (was 1 April 2025)

\emph{For:} Scientific and Statistical Workflow theme issue for \emph{Phil Trans A} as an \emph{Opinion}
% They mention a tex template here: https://royalsocietypublishing.org/rsta/for-authors#question4
% Word length: I should check what they emailed but they say no more than 13 printed pages with 650 per page (whoa, we should be well below that! I am thinking 3-5K would be good)

\begin{abstract}
Increasing biodiversity loss and climate change have led to greater demands for useful ecological models and forecasts. Relevant datasets to meet these demands have also increased in size and complexity, including in their geographical, temporal and phylogenetic scales. While new research often suggests that accounting for these complexities variously increases, removes or otherwise alters major trends, I argue that the fundamental approach to model fitting in ecology makes it impossible to evaluate and compare models. These problems stem in part from continuing gaps between statistical workflows -- where the data processing and model development are often addressed separately from the ecological question and aim -- and scientific workflows, where all steps are integrated. Yet, as ecologists become increasingly computational, and new tools make it easier to share data, the opportunity to close this gap has never been greater. I outline how increased data simulation at multiple steps in the scientific workflow could revolutionize our understanding of ecological systems, yielding new insights. Combining these changes with more open model and data sharing -- and developing new efforts to race the same data -- could be transformative for ecological forecasting. 
\end{abstract}

{\bf Goal:} Increase awareness of how we can merge statistical and scientific workflows in ecology (especially forecasting) and what we would get out of it.

\begin{enumerate}
\item Problem
\begin{enumerate}
\item Climate change, biodiversity crisis etc. has made it critical to understand trends to date and be able to forecast future trends (for policy)
\item Current workflows in ecology are not up to the task
\begin{enumerate}
\item For trends: lots of different stuff reported for seemingly same data/question, makes people outside ecology wonder if we can even well document them, let alone understand them enough to suggest policy
\item For forecasting: also high divergence between models, criticism against process-based approaches (supposedly the most robust?) for lack of transparency (in the model building and calibration) and increasing complexity, with many parameters not supported by data
\end{enumerate}
\item Here we outline the problem and provide a solution!
\end{enumerate}
\item What are current workflows and where are they limiting us?
\begin{enumerate}
\item For trends ...
\begin{enumerate}
\item easy to find different trends through small model tweaks to analyses and/or different data
\item For example, right now many different papers report different biodiversity trends (LPI example?)
\item New workflow would make ecologists understand uncertainty in their model data/combo (and perhaps not see/publish results as so divergent?)
\end{enumerate}
\item For forecasting ... (somehow jump to our focus on process-based models PBMs here? Something like, forecasting is big and there are diverse methods! Near-term iterative, correlative niche models, but PBMs are often considered the gold standard ... mention machine learning?)
\begin{enumerate}
\item as many models as researchers working on process-based models 
\item + accumulation of successive layers in the development of models = significant challenge to scientific transparency, reproducibility, interpretability\\
models often draw inspiration from each other, which is good (way to do science), but not always explicit... (some issues: arbitrarily established parameter value in one model then transmitted to multiple models)
\item focus of researchers: always integrate new mechanisms, new parameters, to increase "realism"...  they intuitively "feel" what kinds of adjustments is needed... but opaque from an external perspective ("black box" of model building and calibration)
\item models rarely fitted as a whole, dozens of parameters without explicitly quantifying parameter uncertainty, and often neglect to propagate this uncertainty
\item simulations of models themselves became a subject of study to disentangle all the processes modelled and understand model sensitivity 
\end{enumerate}
\end{enumerate}
\item Better workflows to the rescue! 
\begin{enumerate}
\item General overview of new workflows
\begin{enumerate}
\item Step 0: Research Qs and hypotheses (with a mechanism) lets you ...
\item Step 1: Build a model!
\item Step 2: Simulate data (and priors or something like priors that forces you to put numbers on stuff)
\item Step 3: Design experiments/data collection (maybe you go back to Step 1 here)
\item Step 4: Simulate data from actual design/collection
\item Step 5: Fit the model to empirical data
\item Step 6: Retrodictive checks (feed back to 0 and 1)
\end{enumerate}
\item New vision of each workflow
\end{enumerate}
\item Conclusion: world is better
\end{enumerate}

Current workflows
\begin{enumerate}
\item 
\end{enumerate}

How much of forecasting do we cover?
\begin{enumerate}
\item PBMs
\item Near-term iterative
\item SDMs
\item Machine learning
\end{enumerate}


Miscellaneous notes/points without a home
\begin{enumerate}
\item Ecologists need to race the same data to make progress for trends and for forecasting (point to make at end of paper maybe? And what is the workflow for this?) ... though LPI is used a lot, perhaps it is  sign that ecology is ready to start racing the same data, but then we need `analysis-ready' data so we're not all slightly differently cleaning the data etc..
\item Machine learning threatens utility of PBMs
\item We need more uncertainty propagation for trends and forecasting (uncertainty is esp. ignored in PBMs)
\item This workflow should lead to less model comparison (AIC, stepwise)
\item This workflow works for machine learning!
\item PBM: workflow should require estimating all parameters together; data simulation should reduce parameter number and highlight non-biol results
\item CUrrent trend workflow: SHould be research questi
\end{enumerate}

\end{document}


\begin{enumerate}
\item
\end{enumerate}

\citep{ospreebbms}

\bibliographystyle{/Users/Lizzie/Documents/EndnoteRelated/Bibtex/styles/besjournals}
\bibliography{/Users/Lizzie/Documents/git/bibtex/LizzieMainMinimal}

\clearpage
\section{Figures}

\begin{figure}[h!]
\includegraphics[width=0.4\textwidth]{..//..//..//..//Professional/images/ColylusVineFruit2.png}
\caption{I am a caption.} 
\label{fig:figname}
\end{figure}